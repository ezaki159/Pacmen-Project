
%
%  $Description: Author guidelines and sample document in LaTeX 2.09$ 
%
%  $Author: ienne $
%  $Date: 1995/09/15 15:20:59 $
%  $Revision: 1.4 $
%

\documentclass[times, 10pt,twocolumn]{article} 
\usepackage{latex8}
\usepackage{times}
\usepackage[noend,linesnumbered,ruled]{algorithm2e}
\usepackage[utf8]{inputenc}

%\documentstyle[times,art10,twocolumn,latex8]{article}

% Pseudocode customization begins here
\SetKwFunction{sleep}{sleep}
\SetKwFunction{calcstate}{calculate\_state}
\SetKwFunction{broadcast}{broadcast}
\SetKwFunction{successor}{is\_successor}
\SetKwFunction{deliver}{deliver}
\SetKwFunction{finaldeliver}{final\_deliver}
% Pseudocode customization ends here

%------------------------------------------------------------------------- 
% take the % away on next line to produce the final camera-ready version 
\pagestyle{empty}

%------------------------------------------------------------------------- 
\begin{document}

\title{DAD-OGP}

\author{Tomás Cunha\\
81201\\ tomas.cunha@tecnico.ulisboa.pt\\
% For a paper whose authors are all at the same institution, 
% omit the following lines up until the closing ``}''.
% Additional authors and addresses can be added with ``\and'', 
% just like the second author.
\and
Guilherme Santos\\
81209\\
guilherme.j.santos@tecnico.ulisboa.pt\\
}

\maketitle
\thispagestyle{empty}

\begin{abstract}
   In modern society, online games have become more and more
   common. In this paper, we introduced a solution for easily
   creating fault-tolerant distributed games in semi-synchronous
   environments (synchronous between servers, asynchronous
   between clients).
   Game clients employ a prediction mechanism to mask delays
   in the server.
   The server may be replicated across an arbitrary number of
   replicas that take its place when a failure is detected.
\end{abstract}



%------------------------------------------------------------------------- 
\Section{Introduction}

With the advent of the internet, multiplayer games
are becoming mainstream. With this in mind, it is important
to design solutions that minimize latency and tolerate faults
so that the player cannot perceive any downtime in the servers.

It is also desirable for the players to communicate with each other
during the course of a game. This means that message order causality must
also be taken into account.

%------------------------------------------------------------------------- 
\Section{Overview}

Our system is based on an abstract server class that employs a round-based
approach to the game cycle. This abstract server implements the communication
logic; every time a new game needs to be added, the developer only needs to
subclass this abstract server and implement the game-specific domain logic.

The same approach is taken on the client: we have an abstract class implementing
the communication logic, and developers wanting to support a new game need only
to implement the game-specific logic.

%------------------------------------------------------------------------- 
\SubSection{Main Game Loop}

At the start of the game, the server waits for all the players
to connect. When a player connects, the server contacts all
currently connected players and announces the new player to them;
this is because players need to know every other player in order
to communicate with them (the chat is implemented in a peer-to-peer
fashion).

When all players have connected, the server calculates the initial
state and sends it to all players.

At this point, the server enters a loop until the game is over. In this
loop, it gathers inputs from the clients, calculates states according to
these inputs, and then sleeps for some amount of time (this time is configured
when the server is started, and corresponds to the round time).

This loop may be represented in pseudocode as follows:

\begin{algorithm}[h]
    \While{running}{
      \sleep{ROUND\_TIME}\;
      \calcstate{}\;
      round\_time := round\_time + 1\;
      \broadcast{state, clients}\;
    }

    \caption{Main game loop}
\end{algorithm}

Out of the functions used in this loop, only \(\textbf{calculate\_state}\) needs
to be implemented by the specific game servers. This massively simplifies
the creation of new games, since the load of managing the distributed side
of the application is taken care of by our abstract class.

%------------------------------------------------------------------------- 
\SubSection{Client-side Overview}

Similarly to the server, there is one abstract class representing a client.
This class uses several abstractions to help make it more extensible. All
the logic for processing received states and sending inputs to the server is
done in this superclass; subclasses need only to implement the logic for
drawing the game board on the screen, based on the states that they receive.

This means that, just like in the server's case, the creation of new types
of games is simple: a developer meaning to create a client for a new type
of game only needs to worry about what sprites to draw in the positions
given to it by the server; anything else is already handled by the abstract
client.

Additionally, the abstract client class takes of care of receiving and sending
chat messages in causal order, meaning that if a message was sent after another
message was received, these two messages will never be received in the wrong
order by a third party who receives both of them (see \ref{chatorder}).


%------------------------------------------------------------------------- 
\Section{Client side concerns}
%------------------------------------------------------------------------- 
\SubSection{Chat message ordering}
\label{chatorder}

To implement causal ordering, we used an implementation of vector clocks as
timestamps to order messages, as described in \cite{timestamps}.

Since the number of clients in our system is known at the start of the game,
each client can create and zero-initialize a vector of scalar timestamps that
is used to order every message that is received. The implementation can be
summarized in the following pseudocode:

\begin{algorithm}[h]
  \If{\successor{m.clock, i.clock}}{
    \deliver{m}\;
  }
  \Else{
    queued\_messages := queued\_messages $\cup$ \{m\}\;
  }

  \caption{When message $m$ is received by client $i$}
\end{algorithm}

In the above pseudocode, a message's clock is considered to be
the successor of the client's successor if, for every timestamp except
for the one of the client that sent that message, the client's timestamp is
greater than or equal to that timestamp, and if the message's timestamp
for the client that sent it is the successor to the corresponding timestamp
on the receiving client's clock.

\begin{algorithm}[h]
  \finaldeliver{m}\;
  to\_send := $\O$\;
  \ForEach{m $\in$ queued\_messages}{
    \If{\successor{m.clock, i.clock}}{
      to\_send := to\_send $\cup$ \{m\}\;
      queued\_messages := queued\_messages $\setminus$ \{m\}\;
    }
  }
  \ForEach{m $\in$ to\_send}{
    \deliver{m}\;
  }

  \caption{When message $m$ is delivered to client $i$}
\end{algorithm}

With these algorithms, we ensure that there will always be a causal ordering
between messages; if a message $m_2$ depends on a message $m_1$, then all
clients will receive $m_1$ before they receive $m_2$.

%------------------------------------------------------------------------ 
\SubSection{State Prediction}

In order to mask away delays in the network between the server and the
client, we employ a prediction mechanism in the client. After a certain
amount of rounds, if the client hasn't received the next state from the
server, it will use the current state it has to simulate the next state.
When the server's state reaches the client, the calculated state will be
corrected if need be.

This way, if the messages between the server and the client suffer some
temporary delay, the user will not immediately perceive this, as the
next state will be calculated on the client's side. Of course, the client
can't guess what movements the remaining players made without the server
telling it, so it might mispredict what the real state actually is.
This misprediction is corrected when the state updates the client with
the real state, but at least the client doesn't see a frozen screen
during this delayed period.

This was done by moving the state calculation logic into the game state class
that is shared between the client and the server; the server is usually the
one calling that calculation method, but if the client perceives a delay in
the connection, it will use that method to keep calculating states.
%------------------------------------------------------------------------- 
\Section{Server side concerns}
%------------------------------------------------------------------------- 
\SubSection{Replication}
In order to ensure that the game keeps going in case of a server failure,
we needed to decide on a replication protocol.

We had several alternatives; one of those would be to use active replication:
every round, the clients would gather states from a majority of replicas and
then draw the state. This solution, however, was not the best for our purposes
for a few reasons:
\begin{itemize}
  \item the client would need to be aware of all the replicas, which
        would introduce some additional unnecessary complexity and overhead.
  \item the client would also need to wait for a majority of responses from the
        servers before drawing the new state, which might be too slow for our
        purposes (since latency is also a concern in this system)
\end{itemize}

For these reasons, active replication was not the correct option, so we opted
for passive replication.

By assuming that the system is synchronous between servers, the problem is
simplified. This assumption is not far-fetched, given the strict time constraints
that game servers are supposed to satisfy. If we assume this, perfect fault
detection is possible: the server maintains a list of replicas, and periodically
send them a message informing them that it is still alive. Whenever there's a
state update, it sends it to all the replicas as well. When the server fails, the
replicas elect a new primary server. Repeat this process until there are no
replicas left. Using this protocol, if we have $f$ replicas (including the
primary server), we can tolerate \(f - 1\) faults.

The only remaining problem, then, is electing a new primary server. This is
trivial, in this case: we can just assign a unique id to each replica (for
example, the IP address and port of that particular replica), and
order the list by that id; then, when a failure is detected, the first
element of that list is selected as the new primary server. If that replica
also failed, it is okay, as this fault will also be quickly detected and the
next one on the list will become the leader.
%------------------------------------------------------------------------- 
\Section{Conclusion}

In a multiplayer game environment, it is of the utmost importance to
reduce latency while preventing the game from being interrupted by a
failure on the server's side. Our proposed solution takes care of the
latter without compromising the former, by employing passive replication
on the server's side along with prediction mechanisms on the client's
side.
%------------------------------------------------------------------------- 
\bibliographystyle{latex8}
\bibliography{latex8}

\end{document}
